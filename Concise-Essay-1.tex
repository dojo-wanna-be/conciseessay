% Options for packages loaded elsewhere
\PassOptionsToPackage{unicode}{hyperref}
\PassOptionsToPackage{hyphens}{url}
%
\documentclass[
]{article}
\usepackage{lmodern}
\usepackage{amssymb,amsmath}
\usepackage{ifxetex,ifluatex}
\ifnum 0\ifxetex 1\fi\ifluatex 1\fi=0 % if pdftex
  \usepackage[T1]{fontenc}
  \usepackage[utf8]{inputenc}
  \usepackage{textcomp} % provide euro and other symbols
\else % if luatex or xetex
  \usepackage{unicode-math}
  \defaultfontfeatures{Scale=MatchLowercase}
  \defaultfontfeatures[\rmfamily]{Ligatures=TeX,Scale=1}
\fi
% Use upquote if available, for straight quotes in verbatim environments
\IfFileExists{upquote.sty}{\usepackage{upquote}}{}
\IfFileExists{microtype.sty}{% use microtype if available
  \usepackage[]{microtype}
  \UseMicrotypeSet[protrusion]{basicmath} % disable protrusion for tt fonts
}{}
\makeatletter
\@ifundefined{KOMAClassName}{% if non-KOMA class
  \IfFileExists{parskip.sty}{%
    \usepackage{parskip}
  }{% else
    \setlength{\parindent}{0pt}
    \setlength{\parskip}{6pt plus 2pt minus 1pt}}
}{% if KOMA class
  \KOMAoptions{parskip=half}}
\makeatother
\usepackage{xcolor}
\IfFileExists{xurl.sty}{\usepackage{xurl}}{} % add URL line breaks if available
\IfFileExists{bookmark.sty}{\usepackage{bookmark}}{\usepackage{hyperref}}
\hypersetup{
  pdftitle={Concise Essay 1},
  pdfauthor={Shaila Ang},
  hidelinks,
  pdfcreator={LaTeX via pandoc}}
\urlstyle{same} % disable monospaced font for URLs
\usepackage[margin=1in]{geometry}
\usepackage{color}
\usepackage{fancyvrb}
\newcommand{\VerbBar}{|}
\newcommand{\VERB}{\Verb[commandchars=\\\{\}]}
\DefineVerbatimEnvironment{Highlighting}{Verbatim}{commandchars=\\\{\}}
% Add ',fontsize=\small' for more characters per line
\usepackage{framed}
\definecolor{shadecolor}{RGB}{248,248,248}
\newenvironment{Shaded}{\begin{snugshade}}{\end{snugshade}}
\newcommand{\AlertTok}[1]{\textcolor[rgb]{0.94,0.16,0.16}{#1}}
\newcommand{\AnnotationTok}[1]{\textcolor[rgb]{0.56,0.35,0.01}{\textbf{\textit{#1}}}}
\newcommand{\AttributeTok}[1]{\textcolor[rgb]{0.77,0.63,0.00}{#1}}
\newcommand{\BaseNTok}[1]{\textcolor[rgb]{0.00,0.00,0.81}{#1}}
\newcommand{\BuiltInTok}[1]{#1}
\newcommand{\CharTok}[1]{\textcolor[rgb]{0.31,0.60,0.02}{#1}}
\newcommand{\CommentTok}[1]{\textcolor[rgb]{0.56,0.35,0.01}{\textit{#1}}}
\newcommand{\CommentVarTok}[1]{\textcolor[rgb]{0.56,0.35,0.01}{\textbf{\textit{#1}}}}
\newcommand{\ConstantTok}[1]{\textcolor[rgb]{0.00,0.00,0.00}{#1}}
\newcommand{\ControlFlowTok}[1]{\textcolor[rgb]{0.13,0.29,0.53}{\textbf{#1}}}
\newcommand{\DataTypeTok}[1]{\textcolor[rgb]{0.13,0.29,0.53}{#1}}
\newcommand{\DecValTok}[1]{\textcolor[rgb]{0.00,0.00,0.81}{#1}}
\newcommand{\DocumentationTok}[1]{\textcolor[rgb]{0.56,0.35,0.01}{\textbf{\textit{#1}}}}
\newcommand{\ErrorTok}[1]{\textcolor[rgb]{0.64,0.00,0.00}{\textbf{#1}}}
\newcommand{\ExtensionTok}[1]{#1}
\newcommand{\FloatTok}[1]{\textcolor[rgb]{0.00,0.00,0.81}{#1}}
\newcommand{\FunctionTok}[1]{\textcolor[rgb]{0.00,0.00,0.00}{#1}}
\newcommand{\ImportTok}[1]{#1}
\newcommand{\InformationTok}[1]{\textcolor[rgb]{0.56,0.35,0.01}{\textbf{\textit{#1}}}}
\newcommand{\KeywordTok}[1]{\textcolor[rgb]{0.13,0.29,0.53}{\textbf{#1}}}
\newcommand{\NormalTok}[1]{#1}
\newcommand{\OperatorTok}[1]{\textcolor[rgb]{0.81,0.36,0.00}{\textbf{#1}}}
\newcommand{\OtherTok}[1]{\textcolor[rgb]{0.56,0.35,0.01}{#1}}
\newcommand{\PreprocessorTok}[1]{\textcolor[rgb]{0.56,0.35,0.01}{\textit{#1}}}
\newcommand{\RegionMarkerTok}[1]{#1}
\newcommand{\SpecialCharTok}[1]{\textcolor[rgb]{0.00,0.00,0.00}{#1}}
\newcommand{\SpecialStringTok}[1]{\textcolor[rgb]{0.31,0.60,0.02}{#1}}
\newcommand{\StringTok}[1]{\textcolor[rgb]{0.31,0.60,0.02}{#1}}
\newcommand{\VariableTok}[1]{\textcolor[rgb]{0.00,0.00,0.00}{#1}}
\newcommand{\VerbatimStringTok}[1]{\textcolor[rgb]{0.31,0.60,0.02}{#1}}
\newcommand{\WarningTok}[1]{\textcolor[rgb]{0.56,0.35,0.01}{\textbf{\textit{#1}}}}
\usepackage{graphicx,grffile}
\makeatletter
\def\maxwidth{\ifdim\Gin@nat@width>\linewidth\linewidth\else\Gin@nat@width\fi}
\def\maxheight{\ifdim\Gin@nat@height>\textheight\textheight\else\Gin@nat@height\fi}
\makeatother
% Scale images if necessary, so that they will not overflow the page
% margins by default, and it is still possible to overwrite the defaults
% using explicit options in \includegraphics[width, height, ...]{}
\setkeys{Gin}{width=\maxwidth,height=\maxheight,keepaspectratio}
% Set default figure placement to htbp
\makeatletter
\def\fps@figure{htbp}
\makeatother
\setlength{\emergencystretch}{3em} % prevent overfull lines
\providecommand{\tightlist}{%
  \setlength{\itemsep}{0pt}\setlength{\parskip}{0pt}}
\setcounter{secnumdepth}{-\maxdimen} % remove section numbering

\title{Concise Essay 1}
\author{Shaila Ang}
\date{30 October, 2020}

\begin{document}
\maketitle

\hypertarget{acupuncture-for-headache}{%
\section{Acupuncture for headache}\label{acupuncture-for-headache}}

A study by Vickers et al., published in the British Medical Journal in
2004, investigated the use of acupuncture for chronic headache in
primary care in a large, randomised trial.\\
The data from this study provides the basis for this concise essay.

This essay provides visualisations of major outcomes considered by the
study at 3 and 12 months. The outcomes of interest are headache severity
score and headache frequency.

\hypertarget{treatments}{%
\section{Treatments}\label{treatments}}

The study measured the effectiveness of acupuncture for chronic headache
in primary care using two treatments labelled `usual care' and
`acupuncture group'. The eligible patients completed a baseline headache
diary for four weeks and were randomised according to ``use
acupuncture'' or ``avoid acupuncture''. During a 3 month period,
patients in the ``use acupuncture'' group received four acupuncture
treatments per month from qualified professionals, and the acupuncture
point prescriptions were unique to each person. This was in addition to
the standard care from their GPs, whereas the ``avoid acupuncture''
group only received standard care and were never referred to acupuncture
specialists.

\begin{verbatim}
## Registered S3 method overwritten by 'pryr':
##   method      from
##   print.bytes Rcpp
\end{verbatim}

\begin{verbatim}
## For best results, restart R session and update pander using devtools:: or remotes::install_github('rapporter/pander')
\end{verbatim}

\begin{verbatim}
## 
## Attaching package: 'dplyr'
\end{verbatim}

\begin{verbatim}
## The following objects are masked from 'package:stats':
## 
##     filter, lag
\end{verbatim}

\begin{verbatim}
## The following objects are masked from 'package:base':
## 
##     intersect, setdiff, setequal, union
\end{verbatim}

\begin{verbatim}
## 
## Attaching package: 'janitor'
\end{verbatim}

\begin{verbatim}
## The following objects are masked from 'package:stats':
## 
##     chisq.test, fisher.test
\end{verbatim}

\hypertarget{age-of-participants}{%
\section{Age of participants}\label{age-of-participants}}

The plot below shows a histogram of the age of the participants in the
study.

\begin{Shaded}
\begin{Highlighting}[]
\NormalTok{plot1 <-}\StringTok{ }\KeywordTok{ggplot}\NormalTok{(acupuncture_for_headache, }\KeywordTok{aes}\NormalTok{(}\DataTypeTok{x =}\NormalTok{ age)) }\OperatorTok{+}\StringTok{ }
\StringTok{      }\KeywordTok{geom_histogram}\NormalTok{(}\DataTypeTok{binwidth=}\DecValTok{5}\NormalTok{)}
\NormalTok{plot1}
\end{Highlighting}
\end{Shaded}

\includegraphics{Concise-Essay-1_files/figure-latex/unnamed-chunk-3-1.pdf}

\hypertarget{headache-frequency-at-3-months}{%
\section{Headache frequency at 3
months}\label{headache-frequency-at-3-months}}

A daily diary was completed by patients over a four week period at three
different stages of the study. The frequency was linked to the severity
in a six-point Likert scale, where 5 equalled the most severe headache
and 0 represented no headache at all. These were recorded 4 times per
day, and this total provided an overall headache score. Additional
health status questionnaires were also completed across the 12 month
period at quarterly intervals, and after 12 months, patients were also
asked to provide an estimate of headache severity on a different scale,
which further provided indications of headache frequency in the cohort.

\begin{Shaded}
\begin{Highlighting}[]
\NormalTok{Theme1 <-}\StringTok{ }\KeywordTok{theme}\NormalTok{(}\DataTypeTok{axis.title.x =} \KeywordTok{element_text}\NormalTok{(}\DataTypeTok{size=}\DecValTok{12}\NormalTok{), }
                \DataTypeTok{axis.text.x  =} \KeywordTok{element_text}\NormalTok{(}\DataTypeTok{size=}\DecValTok{12}\NormalTok{), }
                \DataTypeTok{axis.title.y =} \KeywordTok{element_text}\NormalTok{(}\DataTypeTok{size=}\DecValTok{12}\NormalTok{), }
                \DataTypeTok{axis.text.y  =} \KeywordTok{element_text}\NormalTok{(}\DataTypeTok{size=}\DecValTok{12}\NormalTok{), }
                \DataTypeTok{plot.title =} \KeywordTok{element_text}\NormalTok{(}\DataTypeTok{size =} \DecValTok{16}\NormalTok{),}
                \DataTypeTok{strip.text =} \KeywordTok{element_text}\NormalTok{(}\DataTypeTok{size =} \DecValTok{12}\NormalTok{)) }

\NormalTok{graph1 <-}\StringTok{ }\KeywordTok{ggplot}\NormalTok{(acupuncture_for_headache,}\KeywordTok{aes}\NormalTok{(}\DataTypeTok{x=}\NormalTok{group,  }\DataTypeTok{y=}\KeywordTok{as.numeric}\NormalTok{(f2))) }\OperatorTok{+}
\StringTok{  }\KeywordTok{geom_boxplot}\NormalTok{() }\OperatorTok{+}
\StringTok{  }\KeywordTok{labs}\NormalTok{(}
    \DataTypeTok{x =} \StringTok{"Treatment groups"}\NormalTok{,}
    \DataTypeTok{y =} \StringTok{"Headache frequency at 3 months"}\NormalTok{,}
    \DataTypeTok{title =} \KeywordTok{paste}\NormalTok{(}
      \StringTok{"Headache frequency for both treatment groups at 3 months"}
\NormalTok{    )}
\NormalTok{  ) }\OperatorTok{+}
\StringTok{  }\KeywordTok{scale_y_continuous}\NormalTok{(}\DataTypeTok{breaks =}\NormalTok{ scales}\OperatorTok{::}\KeywordTok{pretty_breaks}\NormalTok{(}\DataTypeTok{n =} \DecValTok{10}\NormalTok{)) }\OperatorTok{+}
\StringTok{  }\NormalTok{Theme1}
\NormalTok{graph1}
\end{Highlighting}
\end{Shaded}

\includegraphics{Concise-Essay-1_files/figure-latex/unnamed-chunk-4-1.pdf}

Graph1 is a boxplot that shows the difference in headache frequency at
three months for acupuncture and usual care groups. Overall, the
acupuncture group experienced lower headache frequency compared to the
usual care group. The median for the acupuncture group was also lower
compared to the usual care group (acupuncture = 10; usual care = 13),
while the interquartile range (IQR) for usual care was higher than the
acupuncture group (acupuncture = 9; usual care = 11.50) (Refer to
Appendix 1). Both graphs are positively skewed. However, a skew of 0.21
in usual care group suggests that it is approximately symmetric, whereas
a skew of 0.67 for acupuncture group suggests that it is moderately
symmetric (Refer to Appendix 1).

\hypertarget{headache-severity-at-12-months}{%
\section{Headache severity at 12
months}\label{headache-severity-at-12-months}}

The headache severity was linked to the frequency and was recorded four
times a day across a four week period to create a baseline, then at 3
months and 12 months after randomisation. The diary used a six-point
Likert scale ranging from `no headache' to `intense incapacitating',
which summed up to give the total headache score. Patients also
completed the SF36 health status questionnaire at baseline, three
months, and one year. Along with additional questionnaires throughout
the study period, after 12 months patients also provided an estimate of
their current and baseline headache severity on a scale of 1-10.

\begin{Shaded}
\begin{Highlighting}[]
\NormalTok{graph2 <-}\StringTok{ }\KeywordTok{ggplot}\NormalTok{(acupuncture_for_headache,}\KeywordTok{aes}\NormalTok{(}\DataTypeTok{x=}\NormalTok{group,  }\DataTypeTok{y=}\KeywordTok{as.numeric}\NormalTok{(pk5))) }\OperatorTok{+}
\StringTok{  }\KeywordTok{geom_boxplot}\NormalTok{() }\OperatorTok{+}
\StringTok{  }\KeywordTok{labs}\NormalTok{(}
    \DataTypeTok{x =} \StringTok{"Treatment groups"}\NormalTok{,}
    \DataTypeTok{y =} \StringTok{"Headache at 12 months"}\NormalTok{,}
    \DataTypeTok{title =} \KeywordTok{paste}\NormalTok{(}
      \StringTok{"Headache severity for both treatment groups at 12 months"}
\NormalTok{    )}
\NormalTok{  ) }\OperatorTok{+}
\StringTok{  }\KeywordTok{scale_y_continuous}\NormalTok{(}\DataTypeTok{breaks =}\NormalTok{ scales}\OperatorTok{::}\KeywordTok{pretty_breaks}\NormalTok{(}\DataTypeTok{n =} \DecValTok{10}\NormalTok{)) }\OperatorTok{+}
\StringTok{  }\NormalTok{Theme1}
\NormalTok{graph2}
\end{Highlighting}
\end{Shaded}

\includegraphics{Concise-Essay-1_files/figure-latex/unnamed-chunk-5-1.pdf}

Graph2 is a boxplot that shows the difference in headache severity at 12
months for acupuncture and usual care groups. In general, the
acupuncture group experienced less severe headaches compared to the
usual care group. The median for the acupuncture group was lower
compared to the usual care (median: acupuncture = 12; usual care = 17),
and the standard deviation for the acupuncture group was 13.72, which
was also lower compared to the usual care group (17.06) (Refer to
Appendix 1). Both graphs are highly skewed (Skewness: acupuncture =
1.76; usual care - 1.59) and have outliers, which needs further
investigation.

\hypertarget{the-relationship-between-headache-frequency-at-12-months-and-baseline}{%
\section{The relationship between headache frequency at 12 months and
baseline}\label{the-relationship-between-headache-frequency-at-12-months-and-baseline}}

\begin{Shaded}
\begin{Highlighting}[]
\NormalTok{Theme2 <-}\StringTok{ }\KeywordTok{theme}\NormalTok{(}\DataTypeTok{axis.title.x =} \KeywordTok{element_text}\NormalTok{(}\DataTypeTok{size=}\DecValTok{10}\NormalTok{), }
                \DataTypeTok{axis.text.x  =} \KeywordTok{element_text}\NormalTok{(}\DataTypeTok{size=}\DecValTok{10}\NormalTok{), }
                \DataTypeTok{axis.title.y =} \KeywordTok{element_text}\NormalTok{(}\DataTypeTok{size=}\DecValTok{8}\NormalTok{), }
                \DataTypeTok{axis.text.y  =} \KeywordTok{element_text}\NormalTok{(}\DataTypeTok{size=}\DecValTok{8}\NormalTok{), }
                \DataTypeTok{plot.title =} \KeywordTok{element_text}\NormalTok{(}\DataTypeTok{size =} \DecValTok{10}\NormalTok{),}
                \DataTypeTok{strip.text =} \KeywordTok{element_text}\NormalTok{(}\DataTypeTok{size =} \DecValTok{10}\NormalTok{)) }


\NormalTok{graph3 <-}\StringTok{ }\KeywordTok{ggplot}\NormalTok{(acupuncture_for_headache, }\KeywordTok{aes}\NormalTok{(}\DataTypeTok{x=}\KeywordTok{as.numeric}\NormalTok{(f5), }\DataTypeTok{y =} \KeywordTok{as.numeric}\NormalTok{(f1))) }\OperatorTok{+}
\StringTok{  }\KeywordTok{geom_point}\NormalTok{() }\OperatorTok{+}
\StringTok{  }\KeywordTok{geom_smooth}\NormalTok{(}\DataTypeTok{method =} \StringTok{"lm"}\NormalTok{, }\DataTypeTok{se =} \OtherTok{TRUE}\NormalTok{)  }\OperatorTok{+}
\StringTok{  }\KeywordTok{facet_grid}\NormalTok{(}\DataTypeTok{cols=}\KeywordTok{vars}\NormalTok{(group)) }\OperatorTok{+}
\StringTok{  }\KeywordTok{labs}\NormalTok{(}
    \DataTypeTok{x=}\StringTok{"Headache frequency at 12 months"}\NormalTok{, }
    \DataTypeTok{y =} \StringTok{"Headache frequency at baseline"}\NormalTok{, }
    \DataTypeTok{title =} \KeywordTok{paste}\NormalTok{(}
    \StringTok{"Headache severity for both treatment groups at 12 months and baseline"}\NormalTok{)) }\OperatorTok{+}
\StringTok{  }\KeywordTok{scale_x_continuous}\NormalTok{(}\DataTypeTok{breaks =}\NormalTok{ scales}\OperatorTok{::}\KeywordTok{pretty_breaks}\NormalTok{(}\DataTypeTok{n =} \DecValTok{10}\NormalTok{)) }\OperatorTok{+}
\StringTok{  }\KeywordTok{scale_y_continuous}\NormalTok{(}\DataTypeTok{breaks =}\NormalTok{ scales}\OperatorTok{::}\KeywordTok{pretty_breaks}\NormalTok{(}\DataTypeTok{n =} \DecValTok{10}\NormalTok{)) }\OperatorTok{+}
\StringTok{  }\NormalTok{Theme1}
\NormalTok{graph3}
\end{Highlighting}
\end{Shaded}

\begin{verbatim}
## `geom_smooth()` using formula 'y ~ x'
\end{verbatim}

\includegraphics{Concise-Essay-1_files/figure-latex/unnamed-chunk-6-1.pdf}

Graph3 is a scatterplot with a regression line that shows the difference
in headache frequency at 12 months and baseline. Generally, there are
not many big differences with the results for both groups. Both
treatment groups have the same median (10), IQR (15), and are both
highly skewed on the baseline (refer to Appendix3.2). At 12 months, the
difference in minimum value and Q1 for both groups is 1, while the
difference in IQR is 2, and maximum value is the same for both groups,
which is 28 (refer to Appendix3.1).

Both graphs showed outliers, which needs further investigation to
understand if acupuncture has a long-term effect on headache.

Appendix

Appendix 1

\begin{Shaded}
\begin{Highlighting}[]
\NormalTok{Appendix1 <-}\StringTok{ }\NormalTok{acupuncture_for_headache }\OperatorTok
\StringTok{  }\KeywordTok{group_by}\NormalTok{(group) }\OperatorTok
\StringTok{  }\KeywordTok{summarize}\NormalTok{(f2)}
\end{Highlighting}
\end{Shaded}

\begin{verbatim}
## `summarise()` regrouping output by 'group' (override with `.groups` argument)
\end{verbatim}

\begin{Shaded}
\begin{Highlighting}[]
\NormalTok{Appendix1}
\end{Highlighting}
\end{Shaded}

\begin{verbatim}
## # A tibble: 301 x 2
## # Groups:   group [2]
##    group          f2
##    <chr>       <dbl>
##  1 acupuncture     8
##  2 acupuncture    24
##  3 acupuncture    11
##  4 acupuncture     2
##  5 acupuncture    10
##  6 acupuncture     8
##  7 acupuncture     3
##  8 acupuncture     9
##  9 acupuncture     8
## 10 acupuncture    23
## # ... with 291 more rows
\end{verbatim}

\begin{Shaded}
\begin{Highlighting}[]
\KeywordTok{descr}\NormalTok{(Appendix1)}
\end{Highlighting}
\end{Shaded}

\begin{verbatim}
## Warning: `funs()` is deprecated as of dplyr 0.8.0.
## Please use a list of either functions or lambdas: 
## 
##   # Simple named list: 
##   list(mean = mean, median = median)
## 
##   # Auto named with `tibble::lst()`: 
##   tibble::lst(mean, median)
## 
##   # Using lambdas
##   list(~ mean(., trim = .2), ~ median(., na.rm = TRUE))
## This warning is displayed once every 8 hours.
## Call `lifecycle::last_warnings()` to see where this warning was generated.
\end{verbatim}

\begin{verbatim}
## Descriptive Statistics  
## f2 by group  
## Data Frame: Appendix1  
## N: 161  
## 
##                     group = acupuncture   group = usual care
## ----------------- --------------------- --------------------
##              Mean                 11.96                13.94
##           Std.Dev                  7.32                 7.60
##               Min                  0.00                 0.00
##                Q1                  7.00                 8.00
##            Median                 10.00                13.00
##                Q3                 16.00                19.50
##               Max                 28.00                28.00
##               MAD                  7.41                 8.15
##               IQR                  9.00                11.25
##                CV                  0.61                 0.55
##          Skewness                  0.67                 0.20
##       SE.Skewness                  0.19                 0.20
##          Kurtosis                 -0.43                -0.82
##           N.Valid                161.00               140.00
##         Pct.Valid                100.00               100.00
\end{verbatim}

Appendix 2

\begin{Shaded}
\begin{Highlighting}[]
\NormalTok{Appendix2 <-}\StringTok{ }\NormalTok{acupuncture_for_headache }\OperatorTok
\StringTok{  }\KeywordTok{group_by}\NormalTok{(group) }\OperatorTok
\StringTok{  }\KeywordTok{summarize}\NormalTok{(pk5)}
\end{Highlighting}
\end{Shaded}

\begin{verbatim}
## `summarise()` regrouping output by 'group' (override with `.groups` argument)
\end{verbatim}

\begin{Shaded}
\begin{Highlighting}[]
\NormalTok{Appendix2}
\end{Highlighting}
\end{Shaded}

\begin{verbatim}
## # A tibble: 301 x 2
## # Groups:   group [2]
##    group         pk5
##    <chr>       <dbl>
##  1 acupuncture  6.25
##  2 acupuncture 51.2 
##  3 acupuncture 25.2 
##  4 acupuncture  1   
##  5 acupuncture  2.5 
##  6 acupuncture 13.5 
##  7 acupuncture  2.75
##  8 acupuncture 19.5 
##  9 acupuncture 21.5 
## 10 acupuncture 38   
## # ... with 291 more rows
\end{verbatim}

\begin{Shaded}
\begin{Highlighting}[]
\KeywordTok{descr}\NormalTok{(Appendix2)}
\end{Highlighting}
\end{Shaded}

\begin{verbatim}
## Descriptive Statistics  
## pk5 by group  
## Data Frame: Appendix2  
## N: 161  
## 
##                     group = acupuncture   group = usual care
## ----------------- --------------------- --------------------
##              Mean                 16.25                22.34
##           Std.Dev                 13.72                17.01
##               Min                  0.00                 0.25
##                Q1                  7.25                10.38
##            Median                 12.00                17.00
##                Q3                 21.50                28.75
##               Max                 73.73                87.25
##               MAD                  8.52                12.97
##               IQR                 14.25                18.31
##                CV                  0.84                 0.76
##          Skewness                  1.76                 1.60
##       SE.Skewness                  0.19                 0.20
##          Kurtosis                  3.46                 2.63
##           N.Valid                161.00               140.00
##         Pct.Valid                100.00               100.00
\end{verbatim}

Appendix 3

\begin{Shaded}
\begin{Highlighting}[]
\NormalTok{Appendix3}\FloatTok{.1}\NormalTok{ <-}\StringTok{ }\NormalTok{acupuncture_for_headache }\OperatorTok
\StringTok{  }\KeywordTok{group_by}\NormalTok{(group) }\OperatorTok
\StringTok{  }\KeywordTok{summarize}\NormalTok{(f5)}
\end{Highlighting}
\end{Shaded}

\begin{verbatim}
## `summarise()` regrouping output by 'group' (override with `.groups` argument)
\end{verbatim}

\begin{Shaded}
\begin{Highlighting}[]
\NormalTok{Appendix3}\FloatTok{.1}
\end{Highlighting}
\end{Shaded}

\begin{verbatim}
## # A tibble: 301 x 2
## # Groups:   group [2]
##    group          f5
##    <chr>       <dbl>
##  1 acupuncture    13
##  2 acupuncture    27
##  3 acupuncture    13
##  4 acupuncture     2
##  5 acupuncture     2
##  6 acupuncture     9
##  7 acupuncture     1
##  8 acupuncture    12
##  9 acupuncture    10
## 10 acupuncture    22
## # ... with 291 more rows
\end{verbatim}

\begin{Shaded}
\begin{Highlighting}[]
\KeywordTok{descr}\NormalTok{(Appendix3}\FloatTok{.1}\NormalTok{)}
\end{Highlighting}
\end{Shaded}

\begin{verbatim}
## Descriptive Statistics  
## f5 by group  
## Data Frame: Appendix3.1  
## N: 161  
## 
##                     group = acupuncture   group = usual care
## ----------------- --------------------- --------------------
##              Mean                 11.38                13.61
##           Std.Dev                  7.47                 7.46
##               Min                  0.00                 1.00
##                Q1                  6.00                 7.00
##            Median                  9.00                12.00
##                Q3                 15.00                18.00
##               Max                 28.00                28.00
##               MAD                  5.93                 7.41
##               IQR                  9.00                11.00
##                CV                  0.66                 0.55
##          Skewness                  0.91                 0.51
##       SE.Skewness                  0.19                 0.20
##          Kurtosis                 -0.06                -0.73
##           N.Valid                161.00               140.00
##         Pct.Valid                100.00               100.00
\end{verbatim}

\begin{Shaded}
\begin{Highlighting}[]
\NormalTok{Appendix3}\FloatTok{.2}\NormalTok{ <-}\StringTok{ }\NormalTok{acupuncture_for_headache }\OperatorTok
\StringTok{  }\KeywordTok{group_by}\NormalTok{(group) }\OperatorTok
\StringTok{  }\KeywordTok{summarize}\NormalTok{(f1)}
\end{Highlighting}
\end{Shaded}

\begin{verbatim}
## `summarise()` regrouping output by 'group' (override with `.groups` argument)
\end{verbatim}

\begin{Shaded}
\begin{Highlighting}[]
\NormalTok{Appendix3}\FloatTok{.2}
\end{Highlighting}
\end{Shaded}

\begin{verbatim}
## # A tibble: 301 x 2
## # Groups:   group [2]
##    group          f1
##    <chr>       <dbl>
##  1 acupuncture    15
##  2 acupuncture    25
##  3 acupuncture    14
##  4 acupuncture    11
##  5 acupuncture     6
##  6 acupuncture     8
##  7 acupuncture     9
##  8 acupuncture    25
##  9 acupuncture     9
## 10 acupuncture    25
## # ... with 291 more rows
\end{verbatim}

\begin{Shaded}
\begin{Highlighting}[]
\KeywordTok{descr}\NormalTok{(Appendix3}\FloatTok{.2}\NormalTok{)}
\end{Highlighting}
\end{Shaded}

\begin{verbatim}
## Descriptive Statistics  
## f1 by group  
## Data Frame: Appendix3.2  
## N: 161  
## 
##                     group = acupuncture   group = usual care
## ----------------- --------------------- --------------------
##              Mean                 15.63                16.15
##           Std.Dev                  6.59                 6.70
##               Min                  3.00                 4.00
##                Q1                 10.00                11.00
##            Median                 15.00                15.00
##                Q3                 20.00                21.00
##               Max                 28.00                28.00
##               MAD                  7.41                 7.41
##               IQR                 10.00                10.00
##                CV                  0.42                 0.41
##          Skewness                  0.47                 0.33
##       SE.Skewness                  0.19                 0.20
##          Kurtosis                 -0.81                -0.93
##           N.Valid                161.00               140.00
##         Pct.Valid                100.00               100.00
\end{verbatim}

\end{document}
